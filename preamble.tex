\documentclass[twocolumn,landscape]{article}

%%%%%%%%%%%%%%%%%%%%%%%%%%%%%%%%%%%%%%%%%%%%%%%%%%%%%%%%%%%%%%%%%%%%%%%%%%%%%%%%
%%%% Pagestyle stuff
%%%%%%%%%%%%%%%%%%%%%%%%%%%%%%%%%%%%%%%%%%%%%%%%%%%%%%%%%%%%%%%%%%%%%%%%%%%%%%%%
\usepackage{fancyhdr}
\pagestyle{fancy}
\fancyhf{}

\usepackage[paperheight=8.5in,%
            paperwidth=11in,%
            outer=.9in,%
            inner=.9in,%
            bottom=.7in,%
            top=.7in,%
            includeheadfoot]{geometry}
\columnsep 3em
\setlength{\headwidth}{9.2in}
\fancyhead[R]{\thepage}
\fancyhead[L]{\sectionname}
\setlength{\headheight}{15.8pt}
\raggedbottom

% Do not enumerate section. This is like using \section*{} except sections
% still turn up in the table of contents.
\setcounter{secnumdepth}{0}

%%%%%%%%%%%%%%%%%%%%%%%%%%%%%%%%%%%%%%%%%%%%%%%%%%%%%%%%%%%%%%%%%%%%%%%%%%%%%%%%
%%%% Packages
%%%%%%%%%%%%%%%%%%%%%%%%%%%%%%%%%%%%%%%%%%%%%%%%%%%%%%%%%%%%%%%%%%%%%%%%%%%%%%%%
\usepackage[utf8]{inputenc}     % For Swedish letters, etc.
\usepackage[swedish]{babel}     % For Swedish avstavning. Heh.
\usepackage{amsmath}		% For symbols
\usepackage{amsthm}             % For theorems
\usepackage{amssymb}	    	% More symbols
\usepackage{tikz}	    	% For graphics
\usepackage[version=3]{mhchem}  % For chemical formulas
\usepackage[colorlinks,%
            linkcolor=black,%
            %pagebackref,%
            bookmarksnumbered=true,%
            linktoc=all]{hyperref}
\usepackage{csquotes}
\usepackage[eprint=false,useprefix=true,sorting=nyt,style=apa,backend=biber]{biblatex}

\DeclareLanguageMapping{swedish}{swedish-apa}

\DefineBibliographyStrings{swedish}{
    references = {Referenser},
    bibliography={Referenser},
}

% APA säger att en url-referens ska se ut t.ex så här:
%
% Vetenskapsrådet. (2011). Samlingen av regler och riktlinjer för forskning.
% Hämtad 9 november 2012, från CODEX: http://www.codex.vr.se/
%
% Skolans apa tycker dock att kolonet efter CODEX istället ska vara ett komma.
% Därför beövs hela harangen här nedan.
\renewbibmacro*{url+urldate}{%
    \ifthenelse{\(\iffieldundef{url}\AND\iffieldundef{abstracturl}\)\OR\NOT\iffieldundef{doi}}
      {}
      {\iffieldundef{abstracturl}
        {}
        {\printtext{\bibcpstring{abstract}}\addspace}%
        \printtext{\bibstring{retrieved}}%
       \setunit{\addspace}%
       \iffieldundef{urlyear}
         {}
         {\printurldate%
          \setunit*{\addcomma\addspace}}%
       \printtext{\bibstring{from}}%
       \setunit*{\addspace}%
       \printfield{urldescription}%
       \setunit*{\addcomma\addspace}%
       \iffieldundef{url}{}{\printfield{url}\renewcommand*{\finentrypunct}{\relax}}%
       \iffieldundef{abstracturl}{}{\printfield{abstracturl}\renewcommand*{\finentrypunct}{\relax}}}}

\addbibresource{biblio.bib}


%%%%%%%%%%%%%%%%%%%%%%%%%%%%%%%%%%%%%%%%%%%%%%%%%%%%%%%%%%%%%%%%%%%%%%%%%%%%%%%%
%%%% Macros
%%%%%%%%%%%%%%%%%%%%%%%%%%%%%%%%%%%%%%%%%%%%%%%%%%%%%%%%%%%%%%%%%%%%%%%%%%%%%%%%
%\newcommand{\sektion}[2]{\stepcounter{section} \renewcommand{\thesection}{#1} \newpage\section{#2} \gdef\sectionname{#1\quad #2, v.~rsinfomonth-rsinfoday}}
% Obsolete commands, just use \section and \subsection as normal.
\newcommand{\sektion}[2]{\stepcounter{section} \renewcommand{\thesection}{#1} \newpage\section{#2} \gdef\sectionname{#1\quad #2}}
\newcommand{\subsektion}[1]{\subsection*{#1} \addcontentsline{toc}{subsection}{#1}}
\newcommand\sectionname{}

% I use this to mark out that I didn't understand, etc.
\newcommand{\sjk}[1]{[[\ensuremath{\bigstar\bigstar\bigstar} #1]]}

\theoremstyle{plain}
\newtheorem{remark}{Anm.}
\newtheorem{patientfall}{Patientfall}

\renewcommand{\labelitemi}{--} % Changes the default bullet style in itemize

\makeatletter
\newenvironment{exercise}[1][]{\begin{trivlist}%
  \item{\bf Övning\@ifnotempty{#1}{ #1}: }\it}{\end{trivlist}}
\newenvironment{solution}{\begin{trivlist}%
  \item{\it Lösning: }}{\end{trivlist}}
\newenvironment{question}[1][]{\begin{trivlist}%
  \item{\bf Fråga\@ifnotempty{#1}{ #1}: }\it}{\end{trivlist}}
\newenvironment{answer}[1][]{\begin{trivlist}%
  \item{\it Svar: }}{\end{trivlist}}
\newenvironment{definition}[1][]{\begin{trivlist}%
  \item{\bf Definition\@ifnotempty{#1}{ #1}: }\it}{\end{trivlist}}
\newenvironment{case}[1][]{\begin{trivlist}%
    \item{\bf Patientfall\@ifnotempty{#1}{ #1}: }\it}{\end{trivlist}}
\newenvironment{warning}[1][]{%
  \begin{trivlist} \item[] \noindent%
  \begingroup\hangindent=2pc\hangafter=-2
  \clubpenalty=10000%
  \hbox to0pt{\hskip-\hangindent\manfntsymbol{127}\hfill}\ignorespaces
  \refstepcounter{equation}\textbf{Varning~\theequation}%
  \@ifnotempty{#1}{\the\thm@notefont \ (#1)}\textbf{.}
  \let\p@@r=\par \def\p@r{\p@@r \hangindent=0pc} \let\par=\p@r}
  {\hspace*{\fill}$\lrcorner$\endgraf\endgroup\end{trivlist}}
\makeatother

\newenvironment{que}{\question}{\endquestion}
\newenvironment{ans}{\answer}{\endanswer}

% The following is for the warning sign.
\DeclareFontFamily{U}{manual}{}
\DeclareFontShape{U}{manual}{m}{n}{ <-> manfnt }{}
\newcommand{\manfntsymbol}[1]{%
    {\fontencoding{U}\fontfamily{manual}\selectfont\symbol{#1}}}
